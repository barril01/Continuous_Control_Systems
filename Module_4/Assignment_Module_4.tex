\documentclass[12pt, letterpaper]{../assignment}
\usepackage{graphicx}
\usepackage{courier}
\usepackage{minted}
\usepackage{amsmath}
\usepackage{commath}
\usepackage{amssymb}
\usepackage{amsfonts} 
\usepackage{cancel}
\usepackage{enumitem}

\usepackage{tikz}
\usetikzlibrary{shapes,arrows,positioning}

\usemintedstyle{monokai}
\oddsidemargin = 0pt
\exercisesheet{Module 4}{Practice Assignment}
\student{Austin Barrilleaux}
\courselabel{EN 525.609}
\semester{Fall 2023}
\usepackage[backend=bibtex,style=numeric,sorting=none]{biblatex}
\bibliography{reference}
\usepackage{color}
\definecolor{light-gray}{rgb}{0.2,0.2,0.2}
\setminted{bgcolor=light-gray}
\setlength{\parindent}{0pt}

\makeatletter
\patchcmd{\minted@colorbg}{\noindent}{\medskip\noindent}{}{}
\apptocmd{\endminted@colorbg}{\par\medskip}{}{}
\makeatother

\begin{document}
\subsection*{Problem 1}
\subsubsection*{Solve the following practice problems in the 9th edition textbook.\\
\begin{itemize}
    \item Chapter 3:
    \begin{itemize}
        \item 10-13 (a) (use MATLAB if / as needed)
    \end{itemize}
\end{itemize}}

\subsubsection*{Given the dynamic equation: {\boldmath
$$ A = \left[\begin{array}{ccc} 0 & 2 & 0 \\ 1 & 2 & 0 \\ -1 & 0 & 1 \end{array}\right],
   B = \left[\begin{array}{c} 0 \\ 1\\ 1 \end{array}\right],
   C = \left[\begin{array}{ccc} 1 & 0 & 1 \end{array}\right] $$}\\
Find the transformation {\boldmath $x(t) =  P \bar{x}(t) $} that transforms the state equations into the controllability canonical form (CCF).}

The characteristic equation of this system is given by:

$$ |sI - A| =  \left|\begin{array}{ccc} s & 2 & 0 \\ 1 & s-2 & 0 \\ -1 & 0 & s-1 \end{array}\right| = 
 (s(s-2)(s-1) + 0 + 0) - (0 + 0 + 2(s-1)) = s^3 -3s^2 + 2 $$

 The $M$ matrix is computed as, where $a_1 = 0$ and $a_2 = -3 $:

 $$ M = \left[\begin{array}{ccc} a_1 & a_2 & 1 \\ a_2 & 1 & 0 \\ 1 & 0 & 0 \end{array}\right]
      = \left[\begin{array}{ccc} 0 & -3 & 1 \\ -3 & 1 & 0 \\ 1 & 0 & 0 \end{array}\right] $$

And $S$ is defined, defined as $ S = [B \ \ AB \ \ A^2 B]$, is:

$$ AB = \left[\begin{array}{ccc} 0 & 2 & 0 \\ 1 & 2 & 0 \\ -1 & 0 & 1 \end{array}\right]
\left[\begin{array}{c} 0 \\ 1\\ 1 \end{array}\right] =  \left[\begin{array}{c} 2 \\ 2\\ 1 \end{array}\right]
$$

$$ A^2 B = \left[\begin{array}{ccc} 0 & 2 & 0 \\ 1 & 2 & 0 \\ -1 & 0 & 1 \end{array}\right]^2
\left[\begin{array}{c} 0 \\ 1\\ 1 \end{array}\right] = 
\left[\begin{array}{ccc} 2 & 4 & 0 \\ 2 & 6 & 0 \\ -1 & -2 & 1 \end{array}\right]
\left[\begin{array}{c} 0 \\ 1\\ 1 \end{array}\right] = \left[\begin{array}{c} 4 \\ 6\\ -1 \end{array}\right]
$$

$\therefore$

$$ S = [B\ \ AB\ \ A^2 B] = \left[\begin{array}{ccc} 0 & 2 & 4 \\ 1 & 2 & 6 \\ 1 & 1 & -1 \end{array}\right] $$

Since the transform is calculated as $P = S M $:

\begin{answer}
  $$  P = \left[\begin{array}{ccc} -2 & 2 & 0 \\ 0 & -1 & 1 \\ -4 & -2 & 1 \end{array}\right] $$
\end{answer}

\subsubsection*{Consider the square matrix, A, below, and:\\
\begin{enumerate}[label=\alph*)]
    \item Determine the characteristic equation (CE)
    \item Find the eigenvalues of A (roots of the CE)
    \item Is this matrix A diagonalizable? Why or why not?
    \item Compute the eigenvectors associated with the computed eigenvalues.
    \item If possible, compute the Diagonal Canonical Form (DCF) of A.
\end{enumerate}
{\boldmath$$  A = \left[\begin{array}{ccc} 1 & -1 \\ 0 & -1 \end{array}\right] $$}}

The characteristic equation is computed as:

$$ |sI - A| = 0 \ \rightarrow \ |sI - A| = \left|\begin{array}{ccc} s-1 & -1 \\ 0 & s+1  \end{array}\right|
= (s-1)(s+1)  = 0 $$

Therefore the characteristic equation is:
\begin{answer}
$$ s^2  - 1 = 0 $$
\end{answer}

The the eigenvalues can be computed using the following equation:

$$ |A - \lambda I| = 0 \ \rightarrow \ |A - \lambda I| =
\left|\begin{array}{ccc} 1-\lambda & -1 \\ 0 & -1-\lambda  \end{array}\right|
= (1-\lambda)(-1-\lambda)  = 0 $$

This simplifies to:

$$ (1-\lambda)(1+\lambda)  = 0 $$

From here, we can see that the eigenvalues are:

\begin{answer}
$$ \lambda = 1, -1 $$
\end{answer}

We could have also determined this by looking at the characteristic equation.
\\\\
\textbf{The matrix A has distinct eigenvalues (all have multiplicity = 1),
therefore the matrix is diagonalizable.}
\begin{answer}
\end{answer}

Eigenvectors are determined by evaluating the following equation:

$$ (\lambda_n I - A)p_n = 0 $$

The eigenvector associated with $\lambda = 1$ is computed as:

$$\left[\begin{array}{ccc} 1-1 & 0-(-1) \\ 0 & 1-(-1) \end{array}\right]
\left[\begin{array}{c} p_{\lambda=1,1} \\ p_{\lambda=1,2} \end{array}\right] =
\left[\begin{array}{c} 0 \\ 0 \end{array}\right]$$

Which simplifies to

$$\left[\begin{array}{ccc} 0 & 1 \\ 0 & 2 \end{array}\right]
\left[\begin{array}{c} p_{\lambda=1,1} \\ p_{\lambda=1,2} \end{array}\right] =
\left[\begin{array}{c} 0 \\ 0 \end{array}\right]$$

The only vector that satisfies this is:

\begin{answer}
    $$ p_{\lambda=1} = \left[\begin{array}{c} 1 \\ 0 \end{array}\right] $$
\end{answer}

The eigenvector associated with $\lambda = -1$ is computed as:

$$\left[\begin{array}{ccc} -1-1 & 0-(-1) \\ 0 & -1-(-1) \end{array}\right]
\left[\begin{array}{c} p_{\lambda=-1,1} \\ p_{\lambda=-1,2} \end{array}\right] =
\left[\begin{array}{c} 0 \\ 0 \end{array}\right]$$

Which simplifies to

$$\left[\begin{array}{ccc} -2 & 1 \\ 0 & 0 \end{array}\right]
\left[\begin{array}{c} p_{\lambda=-1,1} \\ p_{\lambda=-1,2} \end{array}\right] =
\left[\begin{array}{c} 0 \\ 0 \end{array}\right]$$

The only vector that satisfies this is:

\begin{answer}
    $$ p_{\lambda=-1} = \left[\begin{array}{c} 1 \\ 2 \end{array}\right] $$
\end{answer}

Unitized, this vector is:

\begin{answer}
    $$ p_{\lambda=-1} = \left[\begin{array}{c} 1 \\ 2 \end{array}\right]\frac{1}{\sqrt{1+2^2}}
    \approx   \left[\begin{array}{c}  0.4472 \\ 0.8944 \end{array}\right] $$
\end{answer}

Given these eigenvectors, we can show that matrix of eigenvectors, H is:

$$ H = \left[\begin{array}{ccc} 1 & 1 \\ 0 & 2 \end{array}\right] $$

The diagonal of $A$, is $\Lambda$ where:

$$ \Lambda = H^{-1} A H $$

The inverse of H is:

$$ \Lambda = \frac{1}{\text{det}(H)} \left[\begin{array}{ccc} H_{(2,2)} & -H_{(1,2)} \\ -H_{(2,1)} & H_{(1,1)} \end{array}\right]
= \frac{1}{2} \left[\begin{array}{ccc} 2 & -1 \\ 0 & 1 \end{array}\right] $$

Therefore:


$$ \Lambda = H^{-1} A H = \frac{1}{2} \left[\begin{array}{ccc} 2 & -1 \\ 0 & 1 \end{array}\right]
\left[\begin{array}{ccc} 1 & -1 \\ 0 & -1 \end{array}\right]
\left[\begin{array}{ccc} 1 & 1 \\ 0 & 2 \end{array}\right]$$

Which solves to:

$$ \Lambda = H^{-1} A H = \frac{1}{2}
\left[\begin{array}{ccc} 2 & -1 \\ 0 & -1 \end{array}\right]
\left[\begin{array}{ccc} 1 & 1 \\ 0 & 2 \end{array}\right]$$

Then:

$$ \Lambda = H^{-1} A H = \frac{1}{2}
\left[\begin{array}{ccc} 2 & 0 \\ 0 & -2 \end{array}\right]
$$

Finally, the Diagonal Canonical Form of A is:

\begin{answer}
    $$ \Lambda  = \left[\begin{array}{ccc} 1 & 0 \\ 0 & -1 \end{array}\right] $$
\end{answer}

This could have also been determined, by putting the eigenvalues into the form:

\begin{answer}
    $$ \Lambda  = \left[\begin{array}{ccc} \lambda_1 & 0 \\ 0 & \lambda_2 \end{array}\right]
    = \left[\begin{array}{ccc} 1 & 0 \\ 0 & -1 \end{array}\right] $$
\end{answer}

\subsubsection*{Using MATLAB, find the transformation $\bar{x}(t) = H\bar{z}(t)$ so that the state
equations are transformed to DCF if $A$ has distinct eigenvalues,
where system matrices $A$, $B$, and $C$ are below
{\boldmath
$$ A = \left[\begin{array}{ccc} 0 & 2 & 0 \\ 1 & 2 & 0 \\ -1 & 0 & 1 \end{array}\right],
   B = \left[\begin{array}{c} 0 \\ 1\\ 1 \end{array}\right],
   C = \left[\begin{array}{ccc} 1 & 0 & 1 \end{array}\right] $$}}


First in MATLAB, we can calculate the eigenvalues:

\color{white}
\hspace*{6em}\inputminted[frame=leftline,fontsize=\footnotesize]{matlab}
{matlab/compute_eig.m}
\color{black}

This computes the eigenvalues as:

$$ \lambda \approx 1.0000,\ 2.7321,\ -0.7321 $$

The eigenvalues are distinct, so the matrix is diagonalizable.
\\\\
In MATLAB, we can compute H using the \texttt{eig()} function:

\color{white}
\hspace*{6em}\inputminted[frame=leftline,fontsize=\footnotesize]{matlab}
{matlab/hard_way.m}
\color{black}

Where:

\begin{answer}
$$ H = V = \left[\begin{array}{rrr} 0  &  0.5591   & 0.8255\\
                                    0  &  0.7637   & -0.3022\\
                                    1.0000  & -0.3228  &  0.4766 \end{array}\right]  $$
\end{answer}
And the Diagonal of A can be computed as D, where:

$$ \Lambda = D = \left[\begin{array}{rrr} 1  &  0   & 0\\
    0  &  2.7321    &     0\\
    0  & 0 &  -0.7321 \end{array}\right]  $$

Alternatively, $D$ (and subsequently $\Lambda$) can be directly computed as using \texttt{eig()} with:

\color{white}
\hspace*{6em}\inputminted[frame=leftline,fontsize=\footnotesize]{matlab}
{matlab/easy_way.m}
\color{black}

resulting in:

$$ \Lambda = D = \left[\begin{array}{rrr} 1  &  0   & 0\\
    0  &  2.7321    &     0\\
    0  & 0 &  -0.7321 \end{array}\right]  $$
    
\end{document}

